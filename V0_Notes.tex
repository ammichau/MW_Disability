\documentclass[a4paper,11pt]{article}
\usepackage{amsmath,amsfonts, amsthm, amssymb}
\usepackage[margin=1 in]{geometry}
\usepackage{fancyhdr}
\usepackage{graphicx}
\newtheorem{thm}{Theorem}%[section]
\newtheorem{prop}{Proposition}[section]
\newtheorem{cor}[prop]{Corollary}
\newtheorem{lem}[prop]{Lemma}
\theoremstyle{definition}
\newtheorem{defn}{Definition}[section]
\newtheorem{ass}{Assumption}[section]
\theoremstyle{remark}
\newtheorem{rem}[thm]{Remark}
\numberwithin{equation}{section}
%\usepackage{leon}
\setlength{\headheight}{15pt} \pagestyle{fancy} \linespread{1.3}
\fancyhf{}  \rfoot{\thepage}
\bibliographystyle{economet}
\begin{document}
%\delsmieci
\begin{center}
\large
\textbf{Computation Notes}\\
\normalsize
\end{center}
\vspace{5mm}

This file contains notes that accompany programs to calculate $V0(j)$, the expected value of entering an occupation $j\in\{1,2,..J\}$ at birth, before individual types are drawn.
\section{Notes on the Environment}
\begin{itemize}
\item Utility functions and several parameters from Pistaferri \& Low '12.
\begin{eqnarray*}
  u^{w}(c,d) &=& \frac{ce^{\theta d}e^{\eta}}^{1-\gamma}{1-\gamma} \\
  u^{N}(c,d) &=& \frac{ce^{\theta d}}^{1-\gamma}{1-\gamma} 
\end{eqnarray*}
\item Occupations differ in an amplification of "downward" risk only, ie: an economically risky occupation multiplies the transitions to "recession" z and "structural decline" z by a constant factor; a disability risky occupation multiplies the transitions to partial and full disability each by a constant factor as well.
\item I've included asset testing in the SSDI payments. We can easily make this go away later by editing the function.
\end{itemize}
\section{$V^{R}$, Value of Retirement}
The value of retirement is solved as a closed form solution by guess and verify. The trick is to use cash in hand and define the policy as saving a proportion of each period's income.
\begin{eqnarray*}
  V^{R}(d,e,a) &=& \frac{(e^\theta d}(SSI(e)+a))^{1-\gamma}}{1-\gamma}\frac{(1-\delta)^{1-\gamma}}{1-\beta\phi^{T}(\delta R)^{1-\gamma}}\\
  && \text{where} \quad \delta=(1+(\beta\phi^{t}R^{1-\gamma})^{-\frac{1}{\gamma}})^{-1}
\end{eqnarray*}

We can check this at $\beta R=1$, and we find the savings rule is $\delta = \frac{1}{1+R}$, which is what it should be; you would just roll over your savings and eat the interest. 


\end{document}
